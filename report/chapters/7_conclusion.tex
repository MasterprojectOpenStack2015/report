\section[Conclusions and Future Work \texorpdfstring{{\textbf{\tiny \enspace (JE, SK, NK)}}}{}]{Conclusions and Future Work}
\sectionmark{Conclusions and Future Work}
\label{conclusion}

In our masters project we created a platform for the automated installation of an virtual OpenStack environment as well as a framework for some first dependability experiments on this environment. The advantage of our platform is the very fast installation (ten to fifteen minutes) of a complete OpenStack environment on very limited hardware compared to a full bare metal installation. This makes running dependability experiments comparatively easy. Features like snapshotting would not be available on a bare metal setup, but are very useful when repeating such experiments. Further, our platform is easily extendable, both for adding further OpenStack components as well as further experiments.\\

These features can be the foundation for future work. Due to the limited time and personal resources, we were not able to implement the installation of a full OpenStack high availability setup. Such a setup could make it possible to compare the dependability of the ``normal'' OpenStack setup to the high availability one, by running the experiments on both. Further, due to the fact that we use Ansible for the installation, the playbooks themselves can be easily used for installing OpenStack on bare metal. For this, the scripts for configuring the virtual environment could be extended in order to make it possible to configure a bare metal setup. This would allow running the experiments on a bare metal OpenStack installation.